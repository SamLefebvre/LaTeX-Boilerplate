\documentclass{article}

\usepackage{lipsum} % Random text
\usepackage{fullpage} % Use 1 inch margins
\usepackage{color}	% Use colors
\usepackage[dvipsnames]{xcolor} % Colors name
\usepackage{amsmath} % Mathematics symbols
\usepackage{soul} % Highlight
\usepackage{hyperref} % URL
\usepackage{cleveref} % Load cleveref after hyperref
\usepackage{graphicx, graphics, subcaption} % When figures are needed
\usepackage{float} % Improved interface for floating objects
\usepackage[section]{placeins} % ensure all floats for a section appear before the next \section command
\usepackage{enumerate} % Roman numeral

% Configuration
\definecolor{blue(munsell)}{rgb}{0.0, 0.5, 0.69}
\hypersetup{
	linkcolor=violet,
	citecolor=YellowOrange,
	urlcolor=blue(munsell),
	colorlinks = true,
	pageanchor=false, % disable pageanchor for titlepage
}
\graphicspath{{figures/}} % Need two brackets, folder is figures

\begin{document}

\title{This is my \LaTeX{}  boilerplate document}
\author{LastName, FirstName}
\date{July 27 2020}
\def\titledescription{Description of the document}

\makeatletter 
\begin{titlepage}
	\begin{center}
		\vspace*{1cm}
		\Large		
		University Name
		\vspace*{1cm}
		
		\Huge
		\textbf{\@title}
		
		
		\vspace{0.5cm}
		\LARGE
		Subtitle
		
		\vspace{1.5cm}
		
		\textbf{\@author}		
		\vfill		
		\titledescription	
		\vspace{0.8cm}
		
		\Large
		City, \@date
		
	\end{center}
\end{titlepage}
 \makeatother % Insert another document

\input{abstract.tex}

\newpage
\tableofcontents
\newpage

\section{Agenda}

\begin{enumerate}[I.] % Roman numbering and .
	\item \nameref{sec:intro}
	\item \nameref{sec:hello}
	\item \nameref{sec:typesetting_syntax}
	\item \nameref{sec:labeling-and-referencing}
	\item \nameref{sec:math}
	\item \nameref{sec:tables}
	\item \nameref{sec:figures}
\end{enumerate}


\section{Introduction}	
\label{sec:intro}

To use \LaTeX{} these applications are recommended:

\begin{enumerate}
	\item \label{itm:first} Full package of TeXlive \url{https://www.tug.org/texlive/}
	\item MikTeX (in case \ref{itm:first} failed to install):  \url{https://miktex.org/download}
	\item The latest version of TeXstudio: \url{https://www.texstudio.org/}
	\item PDF Reader (any working version): \url{https://get.adobe.com/reader/}
	\item Quick help to install TeXlive and TeXstudio: \url{ https://www.youtube.com/watch?v=82bF7uzgFes}
	\item Better colors[Advanced]: \url{https://tex.stackexchange.com/a/176372}
\end{enumerate}



\section{Hello world}
\label{sec:hello}
Hello world !

\section{Typesetting Syntax}
\label{sec:typesetting_syntax}

\subsection{Paragraph}
\label{subsec:paragraph}
\paragraph{This is my paragraph:} \lipsum[1]

\subsubsection{Sub paragraph}

\subparagraph{This is my subparagraph:} \lipsum[1]

\subsection{Table of contents:} We have generated out table of contents by using the command \verb|\tableofcontents|.

\subsection{Font Effects}
\textbf{Bold text} \verb|\textbf{Bold text}| CTRL+B\\
\textit{Italic text} \verb|\textit{Italic text}| CTRL+I\\
\underline{Underline text} \verb|\underline{Underline text}|\\
\textsc{Small capital text} \verb|\textsc{Small capital text}|\\
\hl{Highlight text} \verb|\hl{Highlight text}|\\
\subsubsection{Font size}

{\tiny 			This is the tiny text size.}\\ % Comments
{\scriptsize 	This is the scriptsize text size}\\
{\footnotesize 	This is the footnotesize text size}\\
{\small 		This is the small text size} \\
{\normalsize 	This is the normalsize text size}\\
{\large 		This is the large text size}\\
{\Large 		This is the Large text size}\\
{\LARGE 		This is the LARGE text size} \\
{\huge 			This is the huge text size}\\
{\Huge 			This is the Huge text size}

\subsubsection{Colors}

Latex support a package for coloring. Just you need to use \verb|\usepackage{color}| for the package and \verb|{\color{color} text}| for the text.\\
\\
{\color{red} This is my red font text.}\\
{\color{green} This is my green font text.}\\
{\color{blue} This is my blue font text.}\\
{\color{yellow} This is my yellow font text.}\\
{\color{cyan} This is my cyan font text.}\\
{\color{magenta} This is my magenta font text.}\\

\section{Labeling and Referencing}
\label{sec:labeling-and-referencing}

Here I would like to call introduction section by using \verb|ref{label}|. Section \ref{sec:intro}. Let us call subsection! In subsection \ref{subsec:paragraph}, we covered paragraph.

\input{mathtyping.tex}

\newpage
\section{Tables}
\label{sec:tables}

Here we are going to deal with the table stuff in \LaTeX.
The very first version of our table can be like this:
\begin{table}[H] % h for here, b for bottom, t for top
	\centering
	\caption{\textsc{This is my first table}}
	\label{tab:myfirsttab}
	\begin{tabular}{c || c} % Two cc reprensents two columns
		\textbf{Notation} & \textbf{Description} \\ \hline \hline
		$x$ & This is the description for the notation\\ \hline
		$y$ & This is the description for the notation\\ \hline
		$z$ & This is the description for the notation\\ \hline
		$w$ & This is the description for the notation\\ \hline
		$\alpha$ & This is the description for the notation\\ \hline
		$\theta$ & This is the description for the notation\\ \hline
		$\beta$ & This is the description for the notation\\ \hline
		$\int_{a}^{b} f(x)$ & B\\ \hline
		$\lambda$ & This notation is related to  Eq. \ref{eq:sample05} \\
		\hline
	\end{tabular}
%	\caption{\textsc{This is my first table}}
\end{table}


\section{Figures}
\label{sec:figures}

Here we will need to have a specific package called \verb|\usepackage{graphicx, graphics, subcaption}|. Then we need to define the path for the figures. We need to do it right after \verb|\begin{document}|
and the command to it is \verb|\graphicspath{{path/}}|

\begin{figure}[ht]
	\centering
	\includegraphics[width=.5\linewidth]{fig1.jpg}
	\caption{Picture of Linear Gradient}
	\label{fig:linearcolors}
\end{figure}

\begin{figure}[ht]
	\centering
	\includegraphics[width=.5\linewidth]{fig2.jpg}
	\caption{Picture of Diamond Gradient}
	\label{fig:diamondcolors}
\end{figure}

I would like to call figure like this. Fig.\ref{fig:linearcolors}.

Calling Fig.\ref{fig:diamondcolors}, named \nameref{fig:diamondcolors}.

\section{Bibliography and Citations}

First, you need to create a bib file. Then, you put you bibtext inside it.
Further, you need to call it before \verb|\end{document}|. Let us call the reference \cite{bibexample}.

\bibliographystyle{apalike}
\bibliography{ref}

\end{document}